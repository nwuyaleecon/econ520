%Jennifer Pan, August 2011

\documentclass[10pt,letter]{article}
	% basic article document class
	% use percent signs to make comments to yourself -- they will not show up.

\usepackage{amsmath}
\usepackage{amssymb}
	% packages that allow mathematical formatting

\usepackage{graphicx}
	% package that allows you to include graphics

\usepackage{setspace}
	% package that allows you to change spacing

\onehalfspacing
	% text become 1.5 spaced

\usepackage{fullpage}
	% package that specifies normal margins


\begin{document}
	% line of code telling latex that your document is beginning


\title{Problem Set 1}

\author{Nicholas Wu}

\date{Fall 2020}
	% Note: when you omit this command, the current dateis automatically included

\maketitle
	% tells latex to follow your header (e.g., title, author) commands.


\section*{Problem 1}

\paragraph{A)} Answer to Problem 1(A) here.

\subparagraph{i)} Answer to Problem 1(A)(i) here.

\subparagraph{ii)} Answer to Problem 1(A)(ii) here.

\subparagraph{iii)} Answer to Problem 1(A)(iii) here.

\paragraph{B)} Answer to Problem 1(B) here.

\subparagraph{i)} Answer to Problem 1(B)(i) here.

\subparagraph{ii)} Answer to Problem 1(B)(ii) here.

\subparagraph{iii)} Answer to Problem 1(B)(iii) here.

\paragraph{C)} Answer to Problem 1(C) here.

\paragraph{D)} Answer to Problem 1(D) here.

\paragraph{E)} Answer to Problem 1(E) here.


\section*{Problem 2}

\paragraph{A)} Answer to Problem 2(A) here.

\paragraph{B)} Answer to Problem 2(B) here.


\section*{Problem 3} Answer to Problem 3 here.


\section*{Problem 4}

\paragraph{A)} Answer to Problem 4(A) here.

\paragraph{B)}  Answer to Problem 4(B) here.


\end{document}
	% line of code telling latex that your document is ending. If you leave this out, you'll get an error
